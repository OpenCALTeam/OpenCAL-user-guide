\chapter{OpenCAL OpenMP version}

With the name OpenCAL-OMP, we identify the OpenMP implementation of
the software library, which can run on all the cores for your CPU. If
you are luky and have a shared memory multiptocessor system,
OpenCAL-OMP can exploit all the cores of all your CPUs. Similarly to
the serial version, OpenCAL-OMP allows for some \emph{unsafe
  operations}, which can significantly speed up your
application. Howevewr, it must be given the utmost attention due to
possible problems related to atomic operations and race condions. In
particular, when many theads perform concurrent operations on the same
memory location(s), if such operation are made by more than one basic
machine operation, it can appen they can interleave , giving rice to
wrong results. Race conditions occur when the logic order of different
operation is not respected. For instance, in a sequence of write-read
operations, it can occur that the read is performed before the write
due to the fact that the thread performing the write is executed
first.

In the following sections, we will introduce OpenCAL-OMP by comparing
examples source code differences with respect to the serial
implemebntations. In the first part of the Chapter, we will deal with
the OpenCAL's safe mode, while in the last one, we will go deep inside
OpenCAL, discussing unsafe operations.


\begin{table}
  \centering
  \begin{tabular}{l|c|c|c|c|c|c}
    \hline
    T version & Serial [s] & 1thr & 2thr & 4thr & 6thr & 8thr\\
    \hline
    \hline
    naive         & 240s & 0.82 (293s) & 1.22 (196s) & 1.53 (157s) & 1.64 (146s) & 1.6 (150s)\\
    active cells  & 23s  & 0.77 (30s)  & 1.36 (17s)  & 1.77 (13s)  & 2.09 (11s)  & 2.3 (10s)\\
    eXtended CA   & 13s  & 0.77 (17s)  & 1.86 (7s)   & 2.6  (5s)   & 2.17  (6s)  & 2.6 (5s)\\
    explicit loop & 12s  & 0.75 (16s)  & 1.2  (10s)  & 2.4  (5s)   & 2.4  (5s)   & 3.0 (4s)\\
    \hline
  \end{tabular}
  \caption{Speedup of the four different
    implementations of the SciddicaS3hex debris flows model accelerated by OpenMP.}
  \label{tab:speedup}
\end{table} 

\begin{table}
  \centering
  \begin{tabular}{l|c|c|c|c|c|c}
    \hline
    S3-hex version & Serial [s] & 1thr & 2thr & 4thr & 6thr & 8thr\\
    \hline
    \hline
    naive         & 1030s & 0.52 (1982s) & 0.9 (1142s) & 1.03 (998s) & 1.13 (913s) & 1.3  (781s)\\
    active cells  & 55s   & 0.86 (64s)   & 1.57 (35s)  & 2.75 (20s)  & 2.5  (22s)  & 3.06 (18s)\\
    eXtended      & 27s   & 0.87 (31s)   & 1.42 (19s)  & 2.7  (10s)  & 2.46 (11s)  & 3.38 (8s)\\
    explicit loop & 16s   & 0.8  (20s)   & 1.33 (12s)  & 2.67 (6s)  & 2.29  (7s)   & 3.2  (5s)\\
    \hline
  \end{tabular}
  \caption{Speedup of the four different
    implementations of the SciddicaS3hex debris flows model accelerated by OpenMP.}
  \label{tab:speedup}
\end{table} 
