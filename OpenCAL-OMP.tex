\chapter{OpenCAL OpenMP version}\label{ch:opencal-omp}

Nowadays, parallel computing is the most effective solution to overcome temporal limits of sequential computation. 
With the name OpenCAL-OMP, we identify the OpenMP implementation of
the software library, that can run on all cores for your CPU. If
you are lucky and have a shared memory multiptocessor system,
OpenCAL-OMP can also exploit all the cores of all your CPUs.

Similarly to the serial version, OpenCAL-OMP allows for some
\emph{unsafe operations}, which can significantly speed up your
application. However, when you use OpenCAL-OMP in \emph{unsafe mode}
you must give the utmost attention to avoid\textsl{ race condition} issues. For
instance, when many threads perform concurrent operations on the same
memory locations and such operations are made by more than one atomic
machine instruction, it can happen that they can interleave, giving rise to
wrong (i.e., non consistent) results. Furthermore, even in the case of atomic operations, the
logic order of execution could not be respected. Thus, for instance, a
read-write logic sequence of atomic operations can actually become a
write-read (wrong) sequence due to the fact that the thread performing
the write operation is executed first.

In the following sections we will introduce OpenCAL-OMP by examples,
highlighting source code the differences with respect to the serial
implementations shown in Chapter \ref{ch:opencal}. In the first part of
the Chapter, we will deal with the OpenCAL's \emph{safe mode}, while in
the last one, we will discuss unsafe operations.

\section{Conway's Game of Life in OpenCAL-OMP}

In Section \ref{sec:cal_life}, we described Conway's Game of Life and
shown a possible implementation using the \verb'OpenCAL' serial
library. Here, we present a \verb'OpenCAL-OMP' implementation of the
same cellular automaton (Listing \ref{lst:calomp_life}), by discussing
the differences with respect its serial implementation (Listing
\ref{lst:cal_life}).

\lstinputlisting[float,floatplacement=H, label=lst:calomp_life, caption=An OpenCAL-OMP implementation of the Conway's game of Life.]{../opencal/OpenCAL-OMP/examples/calomp_life/source/life.c}

As you can see, the OpenMP-based implementation of Life, which uses
only safe operations, is almost identical to the serial one thanks to the seamless parallelization adopted by the library. 
The only differences can be found at lines 3-5 where, instead of including the
OpenCAL header files, you can find the OpenCAL-OMP headers. All the
remaining source code is unchanged. Note that also the OpenCAL-OMP
statements' prefix does not change with respect to the OpenCAL's one
(i.e. \verb'cal' for the functions, \verb'CAL' for the data types, and
\verb'CAL_' for the constants). In practice, if you only use
OpenCAL-OMP in safe mode, besides including the proper OpenCAL-OMP
header files instead of the OpenCAL ones, minimal changes are required to the
serial code.

\section{SciddicaT}

As for the case of Conway's Game of Life, even the OpenCAL-OMP
implementation of the SciddicaT cellular automaton, shown in Lsting
\ref{lst:calomp_sciddicaT}, does not significantly differ from the
serial implementation that you can find in the Section
\ref{sec:sciddicaT}, Listing \ref{lst:cal_sciddicaT}. As before, the
only differences regard the included headers (lines 3-5). Even in this case, as for the Life cellular automaton, due to the fact we
used only OpenCAL-OMP safe operations, mimimal code change is required, besides including the proper
OpenCAL-OMP header files instead of the OpenCAL ones.

\lstinputlisting[label=lst:calomp_sciddicaT, caption=An OpenCAL-OMP implementation of the SciddicaT debris flows simulation model.]{../opencal/OpenCAL-OMP/examples/calomp_sciddicaT/source/sciddicaT.c}

\section{SciddicaT with active cells optimization}
Here we present an OpenCAL-OMP implemenation of SciddicaT, which takes
advantage of the built-in OpenCAL active cells optimization feature. You can
find the complete source code in Listing
\ref{lst:calomp_Sciddicat-activecells}, while the corresponding serial
implementation can be found in Section
\ref{sec:sciddicaT_active}, Listing
\ref{lst:cal_Sciddicat-activecells}.

\lstinputlisting[label=lst:calomp_Sciddicat-activecells, caption=An OpenCAL-OMP implementation of the SciddicaT debris flows simulation model with the active cells optimization.]{../opencal/OpenCAL-OMP/examples/calomp_sciddicaT-activecells/source/sciddicaT.c}

With respect to the Sciddica implementation shown in Listing
\ref{lst:calomp_sciddicaT}, which is exclusively based on safe
OpenCAL-OMP operations, the active cells management as implemented
here requires an unsafe operations. Such unsafe operations are
performed by means of the \verb'calAddActiveCellX2D()' function (line
87), which adds a cell belonging to the neighbourhood to the set $A$
of active cells. As evident, such an operation is considered unsafe because it can
give rise to race condition. In fact, if more threads try to add the
same cell to the set $A$ at the same time, being this a non-atomic
operation, threads' operations can interleave and the outcome be
wrong. In order to avoid this possible error, OpenCAL-OMP is able to
\emph{lock} the memory locations involved in the operations so that
each thread can entirely perform its own task without the risk that other
threads interfere. In order to do this, it is sufficient to place
OpenCAL-OMP in \emph{unsafe} state by calling the
\verb'calSetUnsafe2D()', as done at line 163. No other modifications
to the serial source code are required.

\section{SciddicaT as eXtended CA}
Here we present an OpenCAL-OMP implementation of SciddicaT, which takes
advantage of the built-in unsafe operations. They belong to the
eXtended CA definition and allow for further computational
optimizations. You can find the complete source code of SciddicaT
implemented as an eXtended CA in Listing
\ref{lst:calomp_SciddicaT-unsafe}, while the corresponding serial
implementation can be found in Section \ref{sec:sciddicaT_extended},
Listing \ref{lst:cal_sciddicaT-unsafe}.

\lstinputlisting[label=lst:calomp_SciddicaT-unsafe, caption=An OpenCAL-OMP implementation of the SciddicaT debris flows eXtended CA simulation model with unsafe optimization.]{../opencal/OpenCAL-OMP/examples/calomp_sciddicaT-unsafe/source/sciddicaT.c}

First of all - from a XCA modeling point of view - note that only the topographic altitude and the debris
thickness are now considered as model's substates (lines 25-28,
147-148), as the four outflows substates are no longer
needed. Moreover, the number of elementary process now considered is
two (lines 143-144), instead of three for the previous versions of
SciddicaT.

The call to the \verb'calSetUnsafe2D()' function (line 139) places
OpenCAL-OMP in unsafe mode, allowing to lock memory locations
(i.e. cells) that can be simultaneously accessed by more threads. In
order properly exploit the OpenCAL-OMP's built in lock feature, you
have to use specific functions, which are provided by the
\verb'OpenCAL-OMP/cal2DUnsafe.h' header file (line 6). In the specific
case, besides the already discussed \verb'calAddActiveCellX2D()'
function, the \verb'calAddNext2Dr()' and \verb'calAddNextX2Dr()'
functions are employed (lines 88-89), in place of the combination of
get-set operations, as done in the corresponding serial implementation
(Listing \ref{lst:cal_sciddicaT-unsafe}, lines 84-85). In fact,
consider the source code snippet in Listing \ref{lst:get-set}
(checked out by Listing \ref{lst:cal_sciddicaT-unsafe}). As you can
see, for each not-eliminated cell, the algorithm computes a flow, $f$
(line 5) and then subtracts it from the central cell (line 6), adding
it to the corresponding neighbour (line 7), in order to accomplish mass
balance. In both cases (flow subtraction and adding), a flavor of
\verb'calGet' function is called to read the current value of the
$Q_h$ substate from the next working plane. Subsequently, a flavor of
the \verb'calSet' function is used to update the previously read
value. When a single thread is used to perform such operations, no
race conditions can obviously occur. At the contrary, even in the case
of two concurrent threads, different undesirable situations can take
place, which give rise to a race condition and therefore to a wrong
result. For instance, let's suppose both threads read the value
first, and then write their updated values; in this case, the
resulting value will correspond to the one written by the thread that
writes the value for last, and the contribution of the other thread
will be lost.

\begin{lstlisting}[float,floatplacement=H, label=lst:get-set, caption=Example of non atomic operation made of a combination of get-set calls.]
  // <snip>
  for (n=1; n<sciddicaT->sizeof_X; n++)
  if (!eliminated_cells[n])
  {
    f = (average-u[n])*P.r;
    calSet2Dr (sciddicaT,Q.h,i,j,  calGetNext2Dr (sciddicaT,Q.h,i,j)  -f);
    calSetX2Dr(sciddicaT,Q.h,i,j,n,calGetNextX2Dr(sciddicaT,Q.h,i,j,n)+f);
    // <snip>
  }
  // <snip>
\end{lstlisting}  

In order to avoid such kind of problems when dealing with more
threads, the above mentioned \verb'calAddNext2Dr()' and
\verb'calAddNextX2Dr()' functions lock the cell under consideration
and then perform the get-set operations without the risk other threads
can interfere. In this way, no race conditions can be
triggered. Obviously, there is a side-effect in terms of computational
performance. In fact, as expected, locks can slow down
threads execution and therefore the entire simulation.


\section{SciddicaT with explicit simulation loop}

As for the serial version, also for the OpenMP based release of OpenCAL it is
further possible to improve computational performance of SciddicaT by
avoiding unnecessary substates updating.

As already reported, the \verb'calRun2D()' function used so far to
run the simulation loop updates all the defined substates at the end
of each elementary process. However, in the specific case of the
SciddicaT XCA model, no substates updating should be executed after
the application of the second elementary process, as this just removes
inactive cells from the set $A$.

A new OpenCAL implementation of SciddicaT is presented in Listing
\ref{lst:calomp_sciddicaT-explicit}. It is based on an explicit global
transition function, defined by means of
\verb'calRunAddGlobalTransitionFunc2D()'. It registers a callback
function within which you can both reorder the sequence of elementary
processes to be applied in the generic computational step, and also
select which substates have to be updated after the execution of the
different elementary processes. The SciddicaT implementation here
presented in Listing \ref{lst:calomp_sciddicaT-explicit} also
makes explicit the simulation loop and defines a stopping criterion for
the simulation termination.


\lstinputlisting[label=lst:calomp_sciddicaT-explicit, caption=An OpenCAL-OMP implementation of the SciddicaT XCA debris flows simulation model with explicit simulation loop.]{../opencal/OpenCAL-OMP/examples/calomp_sciddicaT-unsafe-explicit/source/sciddicaT.c}

\section{SciddicaT computational performance}

Table \ref{tab:speedup} resumes computational performance of all the
above illustrated SciddicaT implementations as implemented in
OpenCAL-OMP. The considered case of study is the simulation of the
Tessina landslide shown in Figure \ref{fig:sciddicaT}, which required
a total of 4000 computational steps. The adopted CPU is a Intel Core
i7-4702HQ @ 2.20GHz 4 cores (8 threads) processor, already considered
for the performance evaluation of the corresponding serial SciddicaT
implementations described in Chapter \ref{ch:opencal}. Results are
provided both in terms of elapsed time and speed up with respect to
the corresponding serial version. Elapsed times of the serial
simulations are also reported.

\begin{table}
  \centering
  \footnotesize
  \begin{tabular}{l|c|c|c|c|c|c}
    \hline
    T version & Serial [s] & 1thr & 2thr & 4thr & 6thr & 8thr\\
    \hline
    \hline
    naive         & 240s & 0.82 (293s) & 1.22 (196s) & 1.53 (157s) & 1.64 (146s) & 1.6 (150s)\\
    active cells  & 23s  & 0.77 (30s)  & 1.36 (17s)  & 1.77 (13s)  & 2.09 (11s)  & 2.3 (10s)\\
    eXtended CA   & 13s  & 0.77 (17s)  & 1.86 (7s)   & 2.6  (5s)   & 2.17  (6s)  & 2.6 (5s)\\
    explicit loop & 12s  & 0.75 (16s)  & 1.2  (10s)  & 2.4  (5s)   & 2.4  (5s)   & 3.0 (4s)\\
    \hline
  \end{tabular}
  \caption{Speedup of the four different
    implementations of the SciddicaS3hex debris flows model accelerated by OpenMP.}
  \label{tab:speedup}
\end{table}

As you can see, results are quite good. In particular, the better
results in terms of speed up were obtained for the fully optimized
SciddicaT implementation (with the explicit substate updating
feature), which runs 3 time faster than the corresponding serial
version when executed over 8 threads. Nevertheless, consider that the
SciddicaT simulation model here adopted is quite simple and better
performance in terms of speed up can certainly be obtained for CA
models with more complex transition functions and extended
computational domains.

Eventually, please note how progressive optimizations can considerably
reduce the overall execution time. In fact, if for the naive (i.e., non
optimized at all) serial implementation the elapsed time was 240s, for
the fully optimized parallel version the simulation lasted only 3
seconds, corresponding to a speed up value of 80, i.e. the fully
optimized parallel version runs 80 times faster than the serial naive
implementation.

%% \begin{table}
%%   \centering
%%   \begin{tabular}{l|c|c|c|c|c|c}
%%     \hline
%%     S3-hex version & Serial [s] & 1thr & 2thr & 4thr & 6thr & 8thr\\
%%     \hline
%%     \hline
%%     naive         & 1030s & 0.52 (1982s) & 0.9 (1142s) & 1.03 (998s) & 1.13 (913s) & 1.3  (781s)\\
%%     active cells  & 55s   & 0.86 (64s)   & 1.57 (35s)  & 2.75 (20s)  & 2.5  (22s)  & 3.06 (18s)\\
%%     eXtended      & 27s   & 0.87 (31s)   & 1.42 (19s)  & 2.7  (10s)  & 2.46 (11s)  & 3.38 (8s)\\
%%     explicit loop & 16s   & 0.8  (20s)   & 1.33 (12s)  & 2.67 (6s)  & 2.29  (7s)   & 3.2  (5s)\\
%%     \hline
%%   \end{tabular}
%%   \caption{Speedup of the four different
%%     implementations of the SciddicaS3hex debris flows model accelerated by OpenMP.}
%%   \label{tab:speedup}
%% \end{table} 

%% \begin{table}
%%   \centering
%%   \begin{tabular}{l|c|c|c}
%%     \hline
%%     mbusu version & Serial [s] & OpenMP 6th & OpenCL (Quadro FX 1100M)\\
%%     \hline
%%     \hline
%%     naive         & 7796s & 3.57 (2185s) & 3.52 (2213s)\\
%%     \hline
%%   \end{tabular}
%%   \caption{mbusu Speedup.}
%%   \label{tab:speedup}
%% \end{table} 

%% \begin{table}
%%   \centering
%%   \begin{tabular}{l|c|c}
%%     \hline
%%     Threads & Elapsed time [s] & Speedup\\
%%     \hline
%%     \hline
%%     1       & 7308             & 1     \\
%%     2       &                  &       \\
%%     4       &                  &       \\
%%     6       & 2185             & 3.34  \\
%%     8       &                  &       \\
%%     \hline
%%   \end{tabular}
%%   \caption{mbusu Speedup.}
%%   \label{tab:speedup}
%% \end{table} 


\section{A three-dimensional example}
In Section \ref{sec:mod2}, we described the \emph{mod2} 3D CA and
shown a possible implementation using the \verb'OpenCAL' serial
library. Here, we briefly present a \verb'OpenCAL-OMP' implementation of the
same cellular automaton (Listing \ref{lst:calomp_life}), by discussing
the differences with respect the corresponding serial implementation
(Listing \ref{lst:cal_life}).

\lstinputlisting[float,floatplacement=H, label=lst:calomp_mod2, caption=An OpenCAL-OMP implementation of the mod2 CA.]{../opencal/OpenCAL-OMP/examples/calomp_mod2CA3D/source/mod2CA3D.c}

As you can see, the OpenMP-based implementation, which uses
only safe operations, is almost identical to the serial one. As for the case of the Game of Life CA, the only
differences can be found at lines 3-5 where, instead of including the
OpenCAL header files, we included the OpenCAL-OMP headers. All the
remaining source code is unchanged.

\section{Combining OpenCAL-OMP and OpenCAL-GL and  Global Functions}

As for OpenCAL, it is possible to exploit OpenCAL-GL to have a simple
visualization system by adding few lines of code to your
application. Combining OpenCAL-OMP and OpenCAL-GL does not differ from
what we have done in Section \ref{sec:combining_gl} for OpenCAL and
OpenCAL-GL. Therefore, we remand you to that section for majour
details.

Similarly, you can use the same global reduction functions
described in Section \ref{sec:redution} also in OpenCAL-OMP.  Plese
refer to that section for further details.
