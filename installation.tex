\chapter{Installation} 
\label{ch:installation}



\section{Obtaining \texttt{\ocal}}

\begin{enumerate}
\item  \texttt{\ocal} source code is available on the following \emph{github} repository \url{https://github.com/OpenCALTeam/OpenCAL}. 

\item \texttt{\ocal} is also downloadable as zip file at the following URL: \url{www.urldellozip}
\end{enumerate}




\section{Structure of the Distribution Directory}

The tarball file contains the following files and directories:

\begin{itemize}

	
    \item \textbf{AUTHORS}: Authors of \texttt{\ocal}.
	\item \textbf{\ocal}: core and examples code of the \emph{serial} implementation  
	\item \textbf{\ocal-CL}:  core and examples code of the \emph{Open-CL} implementation  
	\item \textbf{\ocal-GL}:  \texttt{\ocal} graphic core library and examples   
	\item \textbf{\ocal-OMP}:  core and examples code of the \emph{Open-MP}  multicore implementation  

\end{itemize}


\section{Requirements and dependencies}

To compile \texttt{\ocal}, you must have an ANSI C compiler and \texttt{cmake} $\geq$ 2.8 intalled in your system.
\ocal was succesfully compiled and tested with\texttt{gcc} $\geq 4.8$. \texttt{clang} can be also used, taking in mind that it still does not fully support  \emph{Open-MP} natively.
The following is a list of additional dependencies required for each \ocal version:

\begin{itemize}
	\item \texttt{\ocal-OMP}: a C compiler that supports \emph{Open-MP} $\geq 2.0$ (for a list of \emph{OpenMP} compliant compiler see the following link: \url{http://openmp.org/wp/openmp-compilers/})
	\item  \texttt{\ocal-GL}: GLUT/OpenGL libraries and headers. (for example \texttt{freeglut-devel} or \texttt{freeglut3-dev} packages on \texttt{yum/dnf} and Debian-like systems respectively).
	\item \texttt{\ocal-CL}: OpenCL platform should be installed in the system. \texttt{\ocal-CL} was tested on NVIDIA  \emph{OpenCL} implementation (see the following link to know how to obtain the NVIDIA's \texttt{OpenCL} implementation, shipped with the CUDA platform \url{http://docs.nvidia.com/cuda/#installation-guides}).
\end{itemize}



\subsection{CMake}
\texttt{\ocal} uses CMake to generate project files or makefiles for a particular configuration (development environment and library features). If you are on a Unix-like system such as Linux or FreeBSD or have a package system like Fink, MacPorts, Cygwin or Homebrew, you can simply install its CMake package. If not, you can download installers for Windows and OS X from the CMake website.

CMake only generates project files or makefiles that describe how and which characteristics should be compiled. It does not compile the actual \ocal library. To compile \ocal, first generate these files for your chosen development environment and then use them to compile the actual \texttt{\ocal} library.

Suppose you want to compile \texttt{\ocal} and enable support for OpenMP. You will instruct CMake to create the correct makefiles for enabling supportfor OpenMP passing \texttt{-DBUILD\_OPENCAL\_OMP=ON} and \texttt{-DBUILD\_OPENCAL\_OMP\_PARALLEL=ON} as cmake arguments. 

\subsection{Generating MakFiles}
\textbf{This section is about generating the project files or makefiles necessary to compile the}  \texttt{\ocal} \textbf{library, not about compiling the actual library.}

Once you have all necessary dependencies it is time to generate the project files or makefiles for your development environment. CMake needs to know two paths for this: the path to the root directory of the \texttt{\ocal} source tree (i.e. not the src subdirectory) and the target path for the generated files and compiled binaries. If these are the same, it is called an in-tree build, otherwise it is called an out-of-tree build.

We strongly suggest to do an out-of-tree build.
One of several advantages of out-of-tree builds is that you can generate files and compile for different development environments using a single source tree.
To make an out-of-tree build, enter the root directory of the \texttt{\ocal} source tree (i.e. not the src subdirectory) and run CMake using zero or more of the options listed in table \ref{ch:installation:cmakeoptions}  at page  \pageref{ch:installation:cmakeoptions} to control which features will be enabled in the compiled library. The current directory is used as target path, while the path provided as an argument is used to find the source tree.


 \begin{lstlisting}[language=bash,caption={OpenCAL CMake configuration},label={ch:quickstart:simplebuild}]
user@machine:-$ cd opencal && mkdir build && cd build
user@machine:-$ cmake {[-DOPTION]} ../
\end{lstlisting}


\texttt{CMake} options are listed in table \ref{ch:installation:cmakeoptions} alongside their effects and default value.

\begin{table}[]
\centering
\caption{List of CMAKE options}
\label{ch:installation:cmakeoptions}
\begin{tabularx}{\textwidth}{|l|X|l|}
\hline
\textbf{OPTION} & \textbf{EFFECT} & \textbf{DEFAULT}\\ \hline
   \texttt{BUILD\_DOCUMENTATION}  &  Build the HTML based API documentation (Doxygen required)  & OFF   \\ \hline
  \texttt{BUILD\_OPENCAL\_SERIAL} & Build the OpenCAL serial version  & ON   \\ \hline
   \texttt{BUILD\_OPENCAL\_OMP} &  Build the OpenCAL-OMP OpenMP parallel version (OpenMP required)    & OFF \\ \hline
   \texttt{BUILD\_OPENCAL\_OMP\_PARALLEL} &  Controls if OpenCAL-OMP is compiled agaist libomp. If OFF, the OPENMP version uses only one processor! Turn it ON for if you want parallelism  &  ON  \\ \hline
   \texttt{BUILD\_OPENCAL\_CL} &  Build the OpenCAL-CL OpenCL parallel version (OpenCL required)     &OFF\\ \hline
   \texttt{BUILD\_OPENCAL\_GL} & Build the OpenCAL-GL visualization library (OpenGL and GLUT required)      &OFF \\ \hline                          
   \texttt{BUILD\_EXAMPLES} & Build the examples for each OpenCAL version      &\\ \hline
   \texttt{BUILD\_OPENCAL\_PP} &  Build the OpenCAL-C++ version (C++11 compliant compiler Required)    &  OFF\\ \hline
   \texttt{ENABLE\_SHARED} &  Controls whether the library should be compiles as shared  (.so .dll) object (If off static objects will be produced) & OFF\\ \hline                        
\end{tabularx}
\end{table}


\section{Build and installing}
Once \texttt{MakeFile}s have been produced by \texttt{CMake}, everything is set up and ready for compiling. 

\begin{lstlisting}[language=bash,caption={OpenCAL build},label={ch:quickstart:ebuild}]
user@machine:-$ make -jn
\end{lstlisting}
where $n$ is the number of cores of you machine ($-jk$ option enable the parallel compilation using $k$ processors).

You can install the compiled objects (libraries and examples if enabled during the CMake configuration), headers and API documentation in the appropriate folders using the following command:

\begin{lstlisting}[language=bash,caption={OpenCAL installation},label={ch:quickstart:install}]
user@machine:-$ sudo - 
root@machine:-$ make install
root@machine:-$ exit
\end{lstlisting}

or equivalently, if your user is in the \texttt{sudoers} list
\begin{lstlisting}[language=bash,caption={OpenCAL sudo installation},label={ch:quickstart:sudoinstall}]
user@machine:-$ sudo make install
\end{lstlisting}




\section{Web Page and Bug Reporting}

The Web page for \texttt{\ocal} is at 
\url{http://autoti.mat.unical.it} and contains up-to-date news and
a list of bug reports. \ocal's GitHub homepage is at \url{https://github.com/OpenCALTeam/opencal} 
For further information or bug reports contact
\url{mailto:opencal@mat.unical.it} or use the submit an issue at the following url \url{https://github.com/OpenCALTeam/opencal/issues}.

When reporting a bug, please include as much information and
documentation as possible. Helpful information would include
\texttt{\ocal} version, OpenMP/CV implementation and version used,
configuration options, type of computer system, problem
description, and error message output.
