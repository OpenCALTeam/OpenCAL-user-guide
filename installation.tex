\chapter{Installation} 
\label{ch:installation}

\section{Introduction}

This guide presents \verb"OpenCAL", an open source C/C++ library
for implementing models based on the Cellular Automata (CA)
paradighm. Specifically, the library was developed with the aim to permit a
straightforward and simple implementation of Cellular Automata models, which are
particularly suitable for the simulation of spatial extended dynamical systems.
Key features of \verb"OpenCAL" are the
following:

\begin{itemize}
    \item Parallel and Multiplatform. 
    
    \item Support for GPU execution, using OpenCL and CUDA.
    
    \item Support for Complex Cellular Automata 
    
    \item Other Key features here 
\end{itemize}


\section{Obtaining OpenCAL}



\section{Structure of the Distribution Directory}

The distribution contains the following files and subdirectories:

\begin{itemize}

    \item \textbf{AUTHORS}: Authors of \verb"libautoti".


\end{itemize}


\section{Requirements and dependencies}

To compile \verb"libautoti", you must have an ANSI C++ compiler
that includes a full implementation of the Standard Library and
related header files. Additionally, if you want to obtain a
parallel version of the library, you must have an implementation
of the Message Passing Interface (MPI) for the parallel computer
or workstation network you are running on. If you do not have a
native version of MPI for your computer, several
machine-independent implementations are available. Most of the
testing and development of \verb"libautoti" was done by using the
MPICH2 implementation of MPI, which is freely available.
Additional information about MPICH2 is available on the World Wide
Web at \url{http://www.mcs.anl.gov/research/projects/mpich2/}.

\subsection{Installing prerequisites}


\section{Build and installing}

\subsection{cmake options}




\section{Web Page and Bug Reporting}

The World Wide Web page for \verb"libautoti" is
\url{http://autoti.mat.unical.it} and contains up-to-date news and
a list of bug reports. For info or bug reports send an electronic
mail to
\href{mailto:libautoti@mat.unical.it}{libautoti@mat.unical.it}.

When reporting a bug, please include as much information and
documentation as possible. Helpful information would include
\verb"libautoti" version, MPI implementation and version used,
configuration options, type of computer system, problem
description, and error message output.
