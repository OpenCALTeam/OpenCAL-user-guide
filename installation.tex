\chapter{Installation} 
\label{ch:installation}

\section{Introduction}

This guide presents \texttt{\ocal}, an open source C/C++ library
for implementing models based on the Cellular Automata (CA)
paradighm. Specifically, the library was developed with the aim to permit a
straightforward and simple implementation of Cellular Automata models, which are
particularly suitable for the simulation of spatial extended dynamical systems.
Key features of \texttt{OpenCAL} are the
following:

\begin{itemize}
    \item Parallel and Multiplatform. 
    
    \item Support for GPU execution, using OpenCL and CUDA.
    
    \item Support for Complex Cellular Automata 
    
    \item Other Key features here 
\end{itemize}


\section{Obtaining \texttt{\ocal}}

\begin{enumerate}
\item  \texttt{\ocal} source code is available on the following \emph{github} repository \url{https://github.com/OpenCALTeam/OpenCAL}. 

\item \texttt{\ocal} is also downloadable as zip file at the following URL: \url{www.urldellozip}
\end{enumerate}




\section{Structure of the Distribution Directory}

The distribution contains the following files and subdirectories:

\begin{itemize}

	
    \item \textbf{AUTHORS}: Authors of \texttt{\ocal}.
	\item \textbf{\ocal}: core and examples code of the \emph{serial} implementation  
	\item \textbf{\ocal-CL}:  core and examples code of the \emph{Open-CL} implementation  
	\item \textbf{\ocal-GL}:  \texttt{\ocal} graphic core library and examples   
	\item \textbf{\ocal-OMP}:  core and examples code of the \emph{Open-MP}  multicore implementation  

\end{itemize}


\section{Requirements and dependencies}

To compile \texttt{\ocal}, you must have an at least an ANSI C compiler and \texttt{cmake} $\geq$ 2.8 intalled in your system.
\ocal was succesfully compiled and tested with\texttt{gcc} $\geq 4.8$. \texttt{clang} can be also used, taking in mind that it still does not fully support  \emph1{Open-MP} natively.
The following is a list of additional dependencies required for each \ocal version:

\begin{itemize}
	\item \texttt{\ocal-OMP}: a C compiler that supports \emph{Open-MP} $\geq 2.0$ (for a list of \emph{OpenMP} compliant compiler see the following link: \url{http://openmp.org/wp/openmp-compilers/})
	\item  \texttt{\ocal-GL}: GLUT/OpenGL libraries and headers. (for example \texttt{freeglut-devel} or \texttt{freeglut3-dev} packages on \texttt{yum/dnf} and Debian-like systems respectively).
	\item \texttt{\ocal-CL}: 
\end{itemize}



\subsection{Installing prerequisites}


\section{Build and installing}

\subsection{cmake options}




\section{Web Page and Bug Reporting}

The World Wide Web page for \texttt{\ocal} is
\url{http://autoti.mat.unical.it} and contains up-to-date news and
a list of bug reports. For info or bug reports send an electronic
mail to
\href{mailto:libautoti@mat.unical.it}{libautoti@mat.unical.it}.

When reporting a bug, please include as much information and
documentation as possible. Helpful information would include
\texttt{\ocal} version, MPI implementation and version used,
configuration options, type of computer system, problem
description, and error message output.
