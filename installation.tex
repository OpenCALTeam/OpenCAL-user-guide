\chapter{Installation}

\section{Introduction}

This guide presents \verb"libautoti", an open source C++ library
for implementing models based on the Cellular Automata (CA)
approach, a computational science branch for the modelling and
simulation of complex systems. Specifically, the library was
developed with the aim to permit a straightforward and simple
implementation of Macroscopic Cellular Automata models, which are
particularly suitable for the simulation of spatial extended
dynamical systems. Key features of \verb"libautoti" are the
following:

\begin{itemize}
    \item Ability to be linked from C++ programs.
    
    \item Ability to define and manage 1D, 2D and 3D Cellular Automata models (3D models are sequential in this release of the library).

    \item Executable on uniprocessors, multiprocessors, multicomputers, and
    workstation networks thanks to the Message Passing paradigm.

    \item Possibility to integrate with other libraries, e.g. OpenGL or VTK (for simulation visualization) and PGAPack (for models' calibration through Parallel Genetic Algorithms).

    \item Distributed under the Lesser GPL license.
\end{itemize}


\section{Obtaining libautoti}

\verb"libautoti" is available as source code, both for Linux/Unix
and Windows systems, at the URL
\url{http://autoti.mat.unical.it/libautoti/download.html}. The
distribution also contains installation instructions, this guide,
and some examples in C++.

\section{Structure of the Distribution Directory}

The distribution contains the following files and subdirectories:

\begin{itemize}

    \item \textbf{AUTHORS}: Authors of \verb"libautoti".

    \item \textbf{ChangeLog}: Changes new to this release of \verb"libautoti".

    \item \textbf{COPYRIGHT}: Copyright and disclaimer information.

    \item \textbf{NEWS}: News to this release of \verb"libautoti".

    \item \textbf{README}: A file that should be read!

    \item \textbf{configure.ac}: The ``source code'' for the configure script.

    \item \textbf{configure}: Unix shell script to configure Makefile.in for a specific architecture.

    \item \textbf{Makefile.in}: Prototype that is configured into ``Makefile'' by configure.

    \item \textbf{docs}: Directory containing documentation.

    \item \textbf{example}: A directory containing the OpenGLUT versions of Life, both sequential and parallel.

    \item \textbf{sciddica/sciddicaT}: A directory containing the SCIDDICA (T version) Macroscopic Cellular Automata model for debris flows simulation.

    \item \textbf{include}: The \verb"libautoti" include directory.

    \item \textbf{lib}: The directory where the library will be installed in.

    \item \textbf{source}: The source code for the \verb"libautoti" system.

    \item \textbf{test}: A directory containing the Conway's  Game of Life Cellular Automaton (used to verify the
    installation); the subdirectory \verb"sequential" contains the sequential version of the CA, the \verb"parallel"
    sub-directory the parallel one.

\end{itemize}


\section{Requirements}

To compile \verb"libautoti", you must have an ANSI C++ compiler
that includes a full implementation of the Standard Library and
related header files. Additionally, if you want to obtain a
parallel version of the library, you must have an implementation
of the Message Passing Interface (MPI) for the parallel computer
or workstation network you are running on. If you do not have a
native version of MPI for your computer, several
machine-independent implementations are available. Most of the
testing and development of \verb"libautoti" was done by using the
MPICH2 implementation of MPI, which is freely available.
Additional information about MPICH2 is available on the World Wide
Web at \url{http://www.mcs.anl.gov/research/projects/mpich2/}.



\section{Installation Instructions}

The \verb"libautoti" library can be installed into your system as
super user in a common shared directory, e.g.
\verb"/usr/local/libautoti-1.1", or locally as normal user, e.g.
in \verb"$HOME/libautoti-1.1" (where \verb"$HOME" represents the
user's home directory). The installation process is the same for
both cases, except for the fact that the installation directory
changes. In the following we show the installation steps for the
root user, by supposing that the \verb"libautoti-1.1.tar.gz"
archive was saved in the \verb"/root" directory and that the
library will be installed in \verb"/usr/local/libautoti-1.1". Both
sequential and parallel installations are described in the next
two sections.

\subsection{Sequential Installation}

In order to obtain a sequential version of \verb"libautoti",
please follow the steps below:

\begin{enumerate}

    \item mkdir /usr/local/libautoti-1.1

    \item cd /usr/local/libautoti-1.1

    \item tar xovfz /root/libautoti-1.1.tar.gz

    \item ./configure version=sequential

    \item make install

\end{enumerate}

Note that the \verb"configure" script can accept an option,
\verb"CXX", that can be used to specify a different C++ compiler
with respect to that automatically found by the configure script.
So, if you prefer to use the Intel C++ compiler \verb"icc" instead
of the g++ one (which is generally the default C++ compiler in GNU
systems), you can modify the above point 4 as follows:
\begin{itemize}
    \item ./configure CXX=icc version=sequential
\end{itemize}
Obviously, you must have the Intel compiler installed on your
system! In order to test the installation, enter into the
\verb"test/sequential" directory and type the command ./life. You
should obtain the following output:

\begin{verbatim}
   # ./life
   0 0 0 0 0 0 0 0 0 0
   0 0 0 0 0 0 0 0 0 0
   0 0 0 0 0 0 0 0 0 0
   0 0 0 0 0 0 0 0 0 0
   0 0 0 0 0 0 0 0 0 0
   0 0 0 0 0 0 1 1 0 0
   0 0 0 0 0 0 1 0 1 0
   0 0 0 0 0 0 1 0 0 0
   0 0 0 0 0 0 0 0 0 0
   0 0 0 0 0 0 0 0 0 0
\end{verbatim}

\subsection{Parallel Installation}\label{sec:parallel_installation}

In order to obtain a parallel version of \verb"libautoti", please
follow the steps below, which refer to the MPICH2 implementation
of MPI:

\begin{enumerate}

    \item mkdir /usr/local/libautoti-1.1

    \item cd /usr/local/libautoti-1.1

    \item tar xovfz /root/libautoti-1.1.tar.gz

    %\item ./configure mpiinc=\emph{your-mpi-include-directory} mpilib=\emph{your-mpi-library}

    \item ./configure CXX=mpicxx

    \item make install

\end{enumerate}

If you use the MPICH1 implementation of MPI, replace the point 4
with the following:

\begin{itemize}
    \item ./configure mpiinc=\emph{your-mpi-include-directory} mpilib=\emph{your-mpi-library}
\end{itemize}

Note that, the configure script accepts some options. In
particular, \verb"mpiinc" allows to specify the directory in which
the include files of MPI are located, while \verb"mpilib" is the
MPI library. In a typical Linux system having the version 1.2.5 of
the MPICH implementation of MPI installed in the
/usr/local/mpich-1.2.5 directory, the previous point could be the
following:

\begin{itemize}
    \item ./configure mpiinc=/usr/local/mpich-1.2.5/include/ $\backslash
    $ \\ mpilib=/usr/local/mpich-1.2.5/lib/libmpich.a
\end{itemize}

Moreover, the option \verb"CXX" is useful to specify a different
C++ compiler, as before. So, if you prefer to use the Intel C++
compiler \verb icc instead of the g++ ones (which is generally the
default C++ compiler in GNU systems), you can modify the above
point as follows:

\begin{itemize}
    \item ./configure mpiinc=/usr/local/mpich-1.2.5/include/ $\backslash$ \\ mpilib=/usr/local/mpich-1.2.5/lib/libmpich.a $\backslash$ \\CXX=icc
\end{itemize}


In order to test the installation, go to the \verb"test/parallel"
directory and type the command \verb"mpiexec" \verb"-n 2 ./life"
(or \verb"mpirun -np 2 ./life", if you are using MPICH1). You
should obtain the same output of the sequential test (cf. section
above). Note that the command to execute a parallel program
generally depends on the particular MPI implementation and
parallel computer used. In our case, we refer to the MPICH2
(MPICH1) implementation in which a parallel program is executed by
means of the mpiexec (mpirun) script, where the -n (-np) option
allows to specify the number of processes to be executed.


\section{Web Page and Bug Reporting}

The World Wide Web page for \verb"libautoti" is
\url{http://autoti.mat.unical.it} and contains up-to-date news and
a list of bug reports. For info or bug reports send an electronic
mail to
\href{mailto:libautoti@mat.unical.it}{libautoti@mat.unical.it}.

When reporting a bug, please include as much information and
documentation as possible. Helpful information would include
\verb"libautoti" version, MPI implementation and version used,
configuration options, type of computer system, problem
description, and error message output.
