\chapter{Installation} \label{ch:installation}

The 1.0 release of \verb'OpenCAL', here presented, is a collection of
four different software libraries. Under the name \verb'OpenCAL' we
identify the serial version of the library. It comes together with two
different parallel implementations based on OpenMP and OpenCL, namely
OpenCAL-OMP and OpenCAL-CL, respectively. Moreover, OpenCAL-GL
identifies a general purpose OpenGL/GLUT visualization library. Many
examples are also included into the distribution.

The library can be obtained as source code from GitHub to be compiled
in your system. You can chose if compiling all the four different
implementations or only some of them, and also if compling the
examples. Some dependencies must be satisfied, depending on what you
want to compile. Libraries can be compiled both statically (as .lib
files) and as shared objects (as .so files). Eventually, you can
install headers and libraries in your system and also uninstall them
if no longer needed.

In the following Sections we will see all the steps needed to obtain
the software and make it working on your computer.

\section{Requirements and dependencies}

In order to compile OpenCAL, you essentially need CMake (version 2.8
or greater) and an ANSI C compiler (e.g. gcc 4.9 or greater\footnote{
  The clang C compiler can also be used, taking in mind that it still
  does not fully support Open-MP natively.}) installed in your
system. CMake is used to generate the makefiles (or even project files
for several IDEs) to be used to build OpenCAL.

An OpenCL implementation (e.g. by NVIDIA, AMD, Intel) is also needed
to compile OpenCAL-CL. Moreover, still to compile OpenCAL-CL, at least
the 3.1 version of CMake is required. Eventually, a GLUT development
library is needed to compile OpenCAL-GL. The following is a
dependencies list for each of the above software libraries:

\begin{description}
\item[OpenCAL: ] CMake version 2.8 or grater, and a quite new C
  compiler.
\item[OpenCAL-OMP: ] A C compiler supporting at least Open-MP version
  2.0\footnote{For a list of OpenMP compliant compilers see the
    following link: \url{http://openmp.org/wp/openmp-compilers/}.}.
\item[Open-CL: ] CMake version 3.1 or grater and at least one OpenCL
  implementation installed in your system\footnote{OpenCAL-CL was
    tested on the NVIDIA OpenCL implementation. You can find the
    NVIDIA's OpenCL implementation, shipped with the CUDA platform, at
    the following url:
    \url{http://docs.nvidia.com/cuda/\#installation-guides}}.
\item[OpenCAL-GL: ] OpenGL/GLUT headers and libraries \footnote{For
  example \texttt{freeglut-devel} or \texttt{freeglut3-dev} packages
  on \texttt{yum/dnf}- and \texttt{apt}-based systems, respectively.};
\end{description}

\noindent Eventually, Doxygen and Graphviz are required to build the
software documentation, which provides specific information about
library's data structures and functions.

\section{Obtaining OpenCAL}

OpenCAL is available as source code you can downoad from GitHub and
build into your own system. We suggest you to download the stable
software release at the following GitHub url:

\vspace{0.3cm}
\begin{center}
\url{https://github.com/OpenCALTeam/opencal/archive/OpenCAL-1.0.zip}
\end{center}
\vspace{0.3cm}

\noindent Another (not recommended) option is to dowload the
development release. It can be obtained as compressed archive at
\url{https://github.com/OpenCALTeam/opencal/archive/master.zip}, or by
\emph{cloning} the Git repository through the following commands:

\begin{lstlisting}[numbers=none,language=bash]
user@machine:-$ cd <gitwd>
user@machine:-$ git clone https://github.com/OpenCALTeam/OpenCAL
user@machine:-$ cd opencal
\end{lstlisting}

\noindent Here \verb'<gitwd>' represents the directory in your file
system containing your git respositories. If you don't have a git
working directory, we suggest you to create one.


\section{Structure of the Distribution Directory}

The software distribution contains the following files and
directories:

\begin{description}
\item[AUTHORS - ] OpenCAL Authors list;
\item[OpenCAL - ] Core and examples code of the \emph{serial}
  implementation;
\item[OpenCAL-CL - ] Core and examples code of the Open-CL-based
  parallel implementation;
\item[OpenCAL-GL - ] Graphic OpenGL/GLUT-based visualization system
  library and examples;
\item[OpenCAL-OMP - ] Core and examples code of the Open-MP-based
  parallel implementation;
\item[CMakeLists.txt - ] CMake configuration file;
\item[cmake\_uninstall.cmake.in - ] Uninstall script;
\item[configure.cmake - ] File needed by (i.e. included in) CMakeLists.txt;
\item[LICENSE] The GNU LGPL licence;
\item[Other minor files - ] Other minor files can also be included into the distribution.
\end{description}

%% \subsection{CMake}
%% \verb'OpenCAL' uses \verb'CMake' to generate Project Files or
%% Makefiles for a particular configuration (development environment and
%% library features). If you are on a Unix-like system, such as a
%% Linux-based distribution or FreeBSD, or have a package system like
%% Fink, MacPorts, Cygwin or Homebrew, you can simply install its CMake
%% package. If this is not the case, you can download the installers for
%% Windows and Mac OS X from the CMake website.

%% CMake only generates Project Files or Makefiles that describe how and
%% which characteristics should be compiled. It does not compile the
%% actual \verb'OpenCAL' library. To compile \verb'OpenCAL', first
%% generate these files for your chosen development environment and then
%% use them to compile the actual \verb'OpenCAL' library.

%% Suppose you want to compile \verb'OpenCAL' and enable support for
%% OpenMP. You will instruct \verb'CMake' to create the correct Makefiles
%% for enabling support for \verb'OpenMP' using the
%% \texttt{-DBUILD\_OPENCAL\_OMP=ON} and
%% \texttt{-DBUILD\_OPENCAL\_OMP\_PARALLEL=ON} arguments.

\subsection{Generating makfiles}

Once you have satisfied all necessary dependencies it is time to
generate the makefiles (or the project files or for your development
environment) needed to build the software distribution. CMake needs to
know two paths for this:
\begin{enumerate}
\item The path to the OpenCAL source tree root directory;
\item The target path for the generated files and compiled binaries.
\end{enumerate}

If these paths are the same, we are in front of an in-tree build,
otherwise of an out-of-tree build. We strongly suggest to do an
out-of-tree build. One of several advantages of out-of-tree builds is
that you can generate files and compile for different development
environments using a single source tree.

To make an out-of-tree build you need to:
\begin{enumerate}
\item Enter the source tree root directory (e.g. \verb'<gitwd>/opencal/');
\item Create a directory for the binaries
  (e.g. \verb'<gitwd>/opencal/build/') and enter into it;
\item Run CMake using zero or more of the options listed in Table
  \ref{ch:installation:cmakeoptions} to control which features will be
  enabled in the compiled library. The current directory is used as
  target path, while the path provided as an argument is used to find
  the source tree.
\end{enumerate}

If you want to build everything (serial and parallel libraries,
examples and documentation), you can use the following commands:

\begin{lstlisting}[numbers=none,language=bash,label={ch:quickstart:simplebuild}]
  user@machine:-$ cd <gitwd>/opencal/
  user@machine:-$ mkdir build && cd build
  user@machine:-$ cmake ../ -DBUILD_OPENCAL_SERIAL=ON \
                            -DBUILD_OPENCAL_OMP=ON \
                            -DBUILD_OPENCAL_CL=ON \
                            -DBUILD_OPENCAL_GL=ON \
                            -DBUILD_EXAMPLES=ON \
                            -DBUILD_DOCUMENTATION=ON \
                            -DENABLE_SHARED=ON
\end{lstlisting}

Each CMake option corresponds to a target. If you are not interested
to some of them, simply switch off the corresponding option. For
instace, if you set the \verb'ENABLE_SHARED' option to \verb'OFF'
(that, indeed, is the default value), generated makefiles will be set
to build static libraries (.a files under Linux systems) instead of
shared objects (.so files). If you omit a CMake option, the default
value will be assumed (cf. Table \ref{ch:installation:cmakeoptions}).


\begin{table}[]
  \centering
  \caption{List of CMake options, alongside their effects and default
    values}
  \label{ch:installation:cmakeoptions}
  \begin{tabularx}{\textwidth}{lXl}
    \hline
    \textbf{OPTION} & \textbf{EFFECT} & \textbf{DEFAULT}\\
    \hline\hline
    \texttt{BUILD\_OPENCAL\_SERIAL} & Build the OpenCAL serial version  & ON   \\ \hline
    \texttt{BUILD\_OPENCAL\_OMP} &  Build the OpenCAL-OMP OpenMP parallel version (OpenMP required)    & OFF \\ \hline
    %% \texttt{BUILD\_OPENCAL\_OMP\_PARALLEL} &  Controls if OpenCAL-OMP is compiled agaist libomp. If OFF, the OPENMP version uses only one processor! Turn it ON if you want parallelism  &  ON  \\ \hline
    \texttt{BUILD\_OPENCAL\_CL} &  Build the OpenCAL-CL OpenCL parallel version (OpenCL required)     &OFF\\ \hline
    \texttt{BUILD\_OPENCAL\_GL} & Build the OpenCAL-GL visualization library (OpenGL and GLUT required)      &OFF \\ \hline
    \texttt{BUILD\_EXAMPLES} & Build the examples for each OpenCAL version      &\\ \hline
    %% \texttt{BUILD\_OPENCAL\_PP} &  Build the OpenCAL-C++ version (C++11 compliant compiler Required)    &  OFF\\ \hline
    \texttt{BUILD\_DOCUMENTATION}  &  Build the HTML based API documentation (Doxygen and Graphviz required)  & OFF   \\ \hline
    \texttt{ENABLE\_SHARED} &  Controls whether the library should be compiles as shared object (.so). If OFF, static objects (.a) will be produced & OFF\\ \hline
  \end{tabularx}
\end{table}


\section{Build and Install/Uninstall}
Once makefiles have been produced by CMake, everything is set up and
ready for compiling. To build use the following command:

\begin{lstlisting}[numbers=none,language=bash,label={ch:quickstart:ebuild}]
  user@machine:-$ make -jn
\end{lstlisting}
where \verb'n' is the number of cores of your machine you want to use
for speeding up the compilation process.

You can install the compiled objects (libraries and examples - if
enabled during the CMake configuration), headers and API documentation
in the appropriate folders using the following command:

\begin{lstlisting}[numbers=none,language=bash,label={ch:quickstart:install}]
  user@machine:-$ sudo -
  root@machine:-$ make install
\end{lstlisting}

or equivalently, if your user is in the \texttt{sudoers} list

\begin{lstlisting}[numbers=none,language=bash,label={ch:quickstart:sudoinstall}]
  user@machine:-$ sudo make install
\end{lstlisting}

By default, files are installed in \texttt{/usr/local/}folder on \textit{UNIX} and  \texttt{c:/Program Files} on \textit{Windows}.
Installation folder can be configured using the \textit{CMAKE} variable  \texttt{-DCMAKE\_INSTALL\_PREFIX:PATH=\{DIRECTORY\}}.
The followings are the necessary steps for compiling \verb'OpenCAL' and \verb'OpenCAL-OMP' and their examples
in the \texttt{/home/user/libraries}

\begin{lstlisting}[numbers=none,language=bash,label={ch:quickstart:sudoinstall}]
  user@machine:-$ cd ${OPENCAL_ROOT}
  user@machine:-$ mkdir -p build && cd build
  user@machine:-$ cmake -DCMAKE_INSTALL_PREFIX:PATH=/home/user/libraries/ -DBUILD_EXAMPLES=ON -DBUILD_OPENCAL_OMP=ON ..
  user@machine:-$ [sudo] make install
\end{lstlisting}
The following is a sample of the installation output that shows that the installation root is now \texttt{/home/user/libraries/}.
\begin{mdframed}
{\small
\begin{verbatim}
Installing: /home/knotman/libraries/lib/libopencal.a
Installing: /home/knotman/libraries/include/OpenCAL/cal2D.h

...

Installing: /home/user/lib/share/OpenCAL-Examples/OpenCAL/examples/...
\end{verbatim}
}
\end{mdframed}



\begin{lstlisting}[numbers=none,language=bash,label={ch:quickstart:sudoinstall}]

\end{lstlisting}

If you want to uninstall OpenCAL for some reason, you can simply enter the build directory and call make with the uninstall target, as in the following:

\begin{lstlisting}[numbers=none,language=bash,label={ch:quickstart:sudouninstall}]
  user@machine:-$ [sudo] make uninstall
\end{lstlisting}

\section{Web Page and Bug Reporting}

The Web page for OpenCAL is at \url{http://opencal.telesio.unical.it} and
contains up-to-date news and a list of bug reports. \ocal's GitHub
homepage is at \url{https://github.com/OpenCALTeam/opencal}. For
further information or bug reports contact
\url{mailto:opencal@telesio.unical.it} or use the submit an issue at the
following url \url{https://github.com/OpenCALTeam/opencal/issues}.

When reporting a bug, please include as much information and
documentation as possible. Helpful information would include OpenCAL
version, OpenMP/OpenCL implementation and version used,
configuration options, type of computer system, problem description,
and error message output.
