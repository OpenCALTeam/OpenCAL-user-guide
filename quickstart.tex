\chapter{Quick Start}

If you wish to get started by just typing a few lines and running
an example, this section is for you. In any case, more details on
the installation process are given in chapter \nameref{ch:installation} on page
\pageref{ch:installation}.

\section{Download}
OpenCAL source code is available on
\url{https://github.com/OpenCALTeam/opencal}. To obtain a working copy of the
library use the following commands:

\begin{lstlisting}[language=bash,caption={OpenCAL download}]

user@machine:-$ cd <git root>
user@machine:-$ git clone https://github.com/OpenCALTeam/opencal
user@machine:-$ cd opencal
\end{lstlisting}

\section{Build}
\textit{OpenCAL} requires \textit{cmake}\footnote{Minimum required version: 2.8}
and \textit{make} for building the library (see section
\nameref{ch:installation:sect:build} on page \pageref{ch:installation:sect:build} .  

 \begin{lstlisting}[language=bash,caption={OpenCAL download}]

user@machine:-$ cd opencal && mkdir build && cd build
user@machine:-$ cmake ../
user@machine:-$ make
\end{lstlisting}
