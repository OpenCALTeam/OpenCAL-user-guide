\chapter{Quick Start}

If you wish to get started by just typing a few lines and running
an example, this section is for you. In any case, more details on
the installation process are given in the next chapter.

We assume you have downloaded the \verb"libautoti-1.1.tar.gz"
compressed archive into your home directory from the
\verb"libautoti" official website
\url{http://autoti.mat.unical.it}.

To build \verb"libautoti" in \verb"/usr/local/libautoti-1.1" and
run a test example, follow the instructions below as root user.


\section{Sequential version}

\begin{itemize}

    \item mkdir /usr/local/libautoti-1.1

    \item cd /usr/local/libautoti-1.1

    \item tar xovfz \$HOME/libautoti-1.1.tar.gz (\$HOME
represents the user's home directory)

    \item ./configure version=sequential

    \item make install

\end{itemize}


To verify the correctness of the installation, go to the
\verb"test/sequential" directory and execute the \verb"life"
program by typing \verb"./life" at the shell prompt. As a result,
in case of a correct installation, you should obtain the following
output:

\begin{verbatim}
   # ./life
   0 0 0 0 0 0 0 0 0 0
   0 0 0 0 0 0 0 0 0 0
   0 0 0 0 0 0 0 0 0 0
   0 0 0 0 0 0 0 0 0 0
   0 0 0 0 0 0 0 0 0 0
   0 0 0 0 0 0 1 1 0 0
   0 0 0 0 0 0 1 0 1 0
   0 0 0 0 0 0 1 0 0 0
   0 0 0 0 0 0 0 0 0 0
   0 0 0 0 0 0 0 0 0 0
\end{verbatim}

\section{Parallel version}

To build the parallel version of \verb"libautoti", a
message-passing library is required. The following steps refer
to MPICH2
(\url{http://www.mcs.anl.gov/research/projects/mpich2/}), a freely
available, portable implementation of MPI.

\begin{itemize}

    \item mkdir /usr/local/libautoti-1.1

    \item cd /usr/local/libautoti-1.1

    \item tar xovfz \$HOME/libautoti-1.1.tar.gz (\$HOME
represents the user's home directory)

    \item ./configure CXX=mpicxx

    \item make install

\end{itemize}


To verify the installation, go to the \verb"test/parallel"
directory and execute the \verb"life" program by typing
\verb"./mpiexec -n 2 ./life" at the shell prompt. As a result, in
case of a correct installation, you should obtain the same output
as of the sequential test.

The following building steps refer to the standard MPICH1
(\url{http://www-unix.mcs.anl.gov/mpi/mpich1/}) implementation of
MPI.

\begin{itemize}

    \item mkdir /usr/local/libautoti-1.1

    \item cd /usr/local/libautoti-1.1

    \item tar xovfz \$HOME/libautoti-1.1.tar.gz (\$HOME
represents the user's home directory)

    \item ./configure mpiinc=\emph{your-mpi-include-directory} mpilib=\emph{your-mpi-library} (where \emph{your-mpi-include-directory} could
    be /usr/local/mpich-1.2.5/include/, as \emph{your-mpi-library} be /usr/local/mpich-1.2.5/lib/libmpich.a)

    \item make install

\end{itemize}

To verify the installation go to the \verb"test/parallel"
directory and execute the \verb"life" program by typing
\verb"./mpirun -np 2 ./life" at the shell prompt. Even in this
case, you should obtain the same output of the sequential test.

Note that the execution step is completely dependent on the MPI
implementation. These examples use the mpiexec and mpirun scripts
that are distributed with the MPICH2 and MPICH1, respectively.
Other MPI implementations may have other ways to specify the
number of processes to use.
