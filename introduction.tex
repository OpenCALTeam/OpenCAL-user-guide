\chapter{Introduction}

OpenCAL (Open Computing Abstraction Layer) is a parallel computational
software library, developed as an Open Source project at the
Department of Mathematics and Computer Science University of Calabria
(Italy) and released under the LGPL v2.1 license.

OpenCAL allows for the definition of numerical simulation models based
on the Cellular Automata computational paradigm. It also supports
eXtended Cellular Automata (XCA), the Finite Differences method and,
in general, all numerical methods based on uniform computational
grids.

OpenCAL is currently developed in C/C++ and can run in parallel on
both CPUs, thanks to its implementation based on OpenMP, and on GPUs,
thanks to its implementation in OpenCL.

%% Cellular Automata (CA) represent a parallel computing methodology for
%% modelling complex systems. Well known examples of applications include
%% the simulation of natural phenomena such as lava and debris flows,
%% forest fires, agent based social processes such as pedestrian
%% evacuation and highway traffic problems, besides many others (e.g.,
%% theoretical studies).

%% Many Cellular Automata software environments and libraries exist.
%% However, when non-trivial modelling is needed, only non-open source
%% software are generally available. This is particularly true for
%% eXtended Cellular Automata (XCA), adopted for simulating phenomena at
%% a macroscopic point of view, for which only a significant example of
%% non free software exists, namely the CAMELot Cellular Automata
%% Simulation Environment.

%% In order to fill this deficiency in the world of free software, the
%% \verb'OpenCAL' C Library has been developed. Similarly to CAMELot, it
%% allows for a simple and concise definition of both the transition
%% function and the other characteristics of the cellular automaton
%% definition. Moreover, it allows for both sequential and parallel
%% execution, both on CPUs and GPUs (thanks to the adoption of the OpenMP
%% and OpenCL APIs, respectively), hiding most parallel implementation issues
%% to the user.

The library has been tested on both CPUs and GPUs by considering
different Cellular Automata, including the well known Conway's Game of
Life and the SciddicaT XCA debris flows simulation model. Results have
demonstrated the goodness the library both in terms of usability and
performance.

In the present release, 2D and 3D numerical models can be
defined. Actually, even 1D models can be defined as a degenerate case
of 2D CA. The library also offers diverse facilities (e.g. it provides
many predefined cell's neighborhoods), allows to make explicit the
simulation main loop and provides a built in optimization algorithm to
speed up the simulation. Moreover, OpenCAL offers a built in
interactive 2D/3D visualization system developed in OpenGL
Compatibility Profile, so that it can run everywhere, even on old
workstations.

The present manual reports the main usage of the \verb'OpenCAL'
library related to the sequential, OpenMP- and OpenCL-based versions,
the installation procedure, besides examples of application. In
particular, Chapter \ref{ch:installation} deals with download and
installation, while Chapter \ref{ch:CA} introduces the CA and XCA
computational paradigms. The Finite Differences numerical method is
not covered, as it can be considered a particular case of the XCA
paradigm. Chapter \ref{ch:opencal} is about serial CA and XCA
development with OpenCAL, and introduces the different library features
by examples. Chapter \ref{ch:opencal-omp} is about the OpenMP-based
parallel version of OpenCAL and also introduces the library by
examples. Chapter \ref{ch:opencal-cl} briefly introduces to General
Purpose GPU programming with OpenCL and then presents the OpenCL-based
version of OpenCAL, still by examples. OpenCAL-GL is discussed at the
end of each of the above Chapters, together with computational
performances of some of the implemented CA.

%% Eventually, Chapter \ref{ch:utility} ends this user guide by
%% presenting you some usefull library features that weren't presented
%% previously, like reduction functions.
