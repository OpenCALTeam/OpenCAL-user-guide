\chapter{Introduction}

Macroscopic Cellular Automata (MCA) represent a parallel computing
methodology based on the Cellular Automata paradigm for modelling
complex systems at a macroscopic level of description. Well known
examples of applications include the simulation of natural phenomena
such as lava and debris flows, forest fires, agent based social
processes such as pedestrian evacuation and highway traffic problems,
besides many others.

Many Cellular Automata software environments and libraries exist.
However, when non-trivial modelling is needed, only not open source
software are generally available. This is particularly true for
Macroscopic Cellular Automata, for which only a significant example of
non free software exists, namely the CAMELot Cellular Automata
Simulation Environment.

In order to fill this deficiency in the world of free software, the
\verb'OpenCAL' C Library has been developed. Similarly to CAMELot, it
allows for a simple and concise definition of both the transition
function and the other characteristics of the cellular automaton
definition. Moreover, it allows for both sequential and parallel
execution, both on CPUs and GPUs (thanks to the adoption of the OpenMP
and OpenCL, respectively), hiding most parallel implementation issues
to the user.

The library has been tested on both CPUs anf GPUs by considering
different Cellular Automata, including the well known Conway's Game of
Life and the Macroscopic Cellular Automata model SciddicaT for the
simulation of debris flows. Results have demonstrated the goodness the
new library both in terms of usability and performance.

In the present release 1.0 of the library, 2D and 3D cellular automata
can be defined. The library also offers diverse facilities (e.g. it
provides many pre-defined cell's neighborhoods), allows to explicitate
the simulation main cycle and provides a dynamic load balancing
algorithm. Moreover, an interactive 2D/3D visualization system was
developed, based on OpenGL compatibility profile.
